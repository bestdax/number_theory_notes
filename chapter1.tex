\chapter{整除理论}
\section{自然数与整数}
\subsection{基本性质}
这一节里面的内容很基本,暂时略过。
\subsection{最小自然数原理与数学归纳原理}
\begin{axiom}[归纳原(公)理] 设S是N的一个子集,满足条件:
	\begin{axiomenum}
		\item $1 \in S$;
		\item 如果 $n \in S$, 则 $n + 1 \in S$,
	\end{axiomenum}
	那么$S=N$.
\end{axiom}

\begin{theorem}[数学归纳法] 设$P(n)$是关于自然数$n$的一种性质或命题。如果
	\begin{axiomenum}
		\item 当$n=1$时,$P(1)$成立;
		\item 由$P(n)$成立必可推出$P(n+1)$成立,
	\end{axiomenum}
	那么$P(n)$对所有自然数$n$成立。
\end{theorem}

\begin{proof}
	设使$P(n)$成立的所有自然数$n$组成的集合是$S$。$S$是$N$的子集。由条件(i)知$1 \in S$;	由条件(ii), 若$n \in S$,则$n + 1 \in S$。所以由是好归纳原理知$S=N$。
\end{proof}

\begin{theorem}[最小自然数原理]
	设$T$是$N$的一个非空子集。那么,必有$t_0 \in N$,使对任意的$t \in T$有$t_0 \le t$,即$t_0$是$T$中的最小自然数。
\end{theorem}

\begin{theorem}[最大自然数原理]
设$M$是$N$的非空子集。若$M$有上界,即存在$a\in N$,使对任意的$m \in M$有$m \le a$,那么,必有$m_0 \in M$,使对任意的$m
\in M$有$m \le m_0$,即$m_0$是$M$中的最大自然数。
\end{theorem}
