\chapter{整除理论}
\section{自然数与整数}
\subsection{基本性质}
这一节里面的内容很基本,暂时略过。
\subsection{最小自然数原理与数学归纳原理}
\begin{axiom}[归纳原(公)理] 设S是N的一个子集,满足条件:
	\begin{axiomenum}
		\item $1 \in S$;
		\item 如果 $n \in S$, 则 $n + 1 \in S$,
	\end{axiomenum}
	那么$S=N$.
\end{axiom}

\begin{theorem}[数学归纳法] 设$P(n)$是关于自然数$n$的一种性质或命题。如果
	\begin{axiomenum}
		\item 当$n=1$时,$P(1)$成立;
		\item 由$P(n)$成立必可推出$P(n+1)$成立,
	\end{axiomenum}
	那么$P(n)$对所有自然数$n$成立。
\end{theorem}

\begin{proof}
	设使$P(n)$成立的所有自然数$n$组成的集合是$S$。$S$是$N$的子集。由条件(i)知$1 \in S$;	由条件(ii), 若$n \in S$,则$n + 1 \in S$。所以由归纳原理知$S=N$。
\end{proof}

\begin{theorem}[最小自然数原理]\label{thrm:minint}
	设$T$是$N$的一个非空子集。那么,必有$t_0 \in N$,使对任意的$t \in T$有$t_0 \le t$,即$t_0$是$T$中的最小自然数。
\end{theorem}

\begin{proof}
	考虑由所有这样的自然数$s$组成的集合$S$:对任意的$t \in T$必有$s \le t$。由于$1$满足这样的条件,所以$1 \in
		S$,$S$非空。此外,若$t_1 \in T$(因$T$非空所以必有$t_1$),则$t_1 + 1 > t_1$,所以$t_1 + 1 \notin
		S$。由这两点及归纳原理就推出:必有$s_0 \in S$使得$s_0 + 1 \notin S$(为什么)。我们来证明必有$s_0 \in
		T$。因若不然,则对任意的$t \in T$必有$t>s_0$,因而$t \ge s_0 + 1$。这表明$s_0 + 1 \in S$,矛盾。取$t_0 =
		s_0$就证明了定理\ref{thrm:minint}。
\end{proof}

\begin{theorem}[最大自然数原理]\label{thrm:maxint}
	设$M$是$N$的非空子集。若$M$有上界,即存在$a\in N$,使对任意的$m \in M$有$m \le a$,那么,必有$m_0 \in M$,使对任意的$m
		\in M$有$m \le m_0$,即$m_0$是$M$中的最大自然数。
\end{theorem}

\begin{proof}
	考虑由所有这样的自然数$t$组成的集合$T$:对任意的$m \in M$有$m \le t$。由条件知$a \in
	T$,所以$T$非空。由定理\ref{thrm:minint}知,集合$T$中有最小自然数$t_0$。我们来证明$t_0 \in
	M$。若不然,则对任意的$m\in M$必有$m < t_0$,因而$m\le t_0 -1$。这样就推出$t_0 -1\in
	T$,但这和$t_0$的最小性矛盾。取$m_0 = t_0$就证明了定理\ref{thrm:maxint}。
\end{proof}
